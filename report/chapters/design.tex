\chapter{Design}
% MoSCoW requirements

F = Functional

NF = Non-Functional

M = Must

S = Should

C = Could

\begin{center}
	\begin{xltabular}{0.82\paperwidth}{ | p{1.5cm} | p{2cm} | X | } 
		\hline
		F / NF & M / S / C & Requirement Description \\ \hline
		\multicolumn{3}{|c|}{Data Processing} \\ \hline
		F & M & The system must allow users to filter datasets according to custom criteria. \\ \hline
		F & M & The system must allow users to join two datasets together according to custom criteria. \\ \hline
		F & M & The system must allow users to group portions of datasets together by unique keys. \\ \hline
		F & M & The system must allow users to apply custom functions to each row of a dataset. \\ \hline
		F & M & The system must allow users to apply a reducer function to a dataset. \\ \hline 
		NF & S & The complexities of the system should be hidden from the user; the operation of the system should be identical whether the user is running the code locally or over a cluster. \\ \hline
		\multicolumn{3}{|c|}{Cluster} \\ \hline
		F & M & The user should be able to index specific columns of the data to allow these rows to be accessed more quickly by the cluster. \\ \hline
		F & S & The cluster should be able to handle node failures by restarting them and requesting new work. \\ \hline
		NF & S & The cluster should feature some form of load balancing to ensure that nodes are not idling if there is still work to be performed. \\ \hline
		\multicolumn{3}{|c|}{Main Nodes} \\ \hline
		F & M & The main node must delegate work to the cluster nodes to perform the calculation efficiently. \\ \hline
		F & M & The main node must collect the results from the cluster nodes to produce the final output for the user. \\ \hline
		F & M & The main node must handle cluster node failures to ensure that the calculation completes successfully anyway. \\ \hline
		\multicolumn{3}{|c|}{Automated ETL Workflows} \\ \hline
		F & S & The user should be able to define workflows, defined as the below requirements in this section \textit{(Automated ETL)}, using some kind of configuration file. \\ \hline
		F & S & The system should receive files sent from the user, ingest them into the system automatically, and start a calculation when the files are sent. \\ \hline
		F & S & The user should start calculations manually, schedule them for a specific time, or schedule them at a regular interval. \\ \hline
		F & S & The system should output the results of a calculation to a user-defined location. \\ \hline
		F & S & The system should raise alerts when an unrecoverable failure occurs during a calculation. \\ \hline
		F & C & The system could send alerts via email to the user when an unrecoverable failure occurs during a calculation. \\ \hline
		F & C & The system could monitor a specific folder location for changes, then initiate a calculation when files in that location are changed. \\ \hline
		\multicolumn{3}{|c|}{Dashboard} \\ \hline
		F & S & The system should have a dashboard to display basic information about user-created workflows \textit{(from Automated ETL)}, including source and destination files, and previous executions. \\ \hline
		F & C & The system could allow the user to start or schedule workflows from the dashboard. \\
		\hline
	\end{xltabular}
\end{center}

% Choice of language
% Python front-end - ease of use
% Scala back-end - why? over other OO languages like Java, JavaScript, C, Rust even non-OO languages

% Runtime
% Containerisation - Docker
% Kubernetes / Docker Swarm

% Database backend
% Custom implementation vs existing - chose existing due to time constraints
% Choice of existing - cassandra vs other solutions like SQL, NoSQL
% - Partitioning